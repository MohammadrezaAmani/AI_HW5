%% -!TEX root = AUTthesis.tex
% در این فایل، عنوان پایان‌نامه، مشخصات خود، متن تقدیمی‌، ستایش، سپاس‌گزاری و چکیده پایان‌نامه را به فارسی، وارد کنید.
% توجه داشته باشید که جدول حاوی مشخصات پروژه/پایان‌نامه/رساله و همچنین، مشخصات داخل آن، به طور خودکار، درج می‌شود.
%%%%%%%%%%%%%%%%%%%%%%%%%%%%%%%%%%%%
% دانشکده، آموزشکده و یا پژوهشکده  خود را وارد کنید
\faculty{دانشکده ریاضی و علوم کامپیوتر}
% گرایش و گروه آموزشی خود را وارد کنید
\department{رشته علوم کامپیوتر}
% عنوان پایان‌نامه را وارد کنید
\fatitle{
طبقه بندی و خوشه بندی داده
	\\[.75 cm]
	گزارش تمرین سری پنجم}
% نام استاد(ان) راهنما را وارد کنید
\firstsupervisor{دکتر مهدی قطعی}
%\secondsupervisor{استاد راهنمای دوم}
% نام استاد(دان) مشاور را وارد کنید. چنانچه استاد مشاور ندارید، دستور پایین را غیرفعال کنید.
%\firstadvisor{دکتر مهدی قطعی}
%\secondadvisor{استاد مشاور دوم}
% نام نویسنده را وارد کنید
\name{محمدرضا}
% نام خانوادگی نویسنده را وارد کنید
\surname{امانی}
%%%%%%%%%%%%%%%%%%%%%%%%%%%%%%%%%%
% \thesisdate{\date}

% چکیده پایان‌نامه را وارد کنید
\fa-abstract{
مجموعه داده شامل بیش از 284807 داده از داده‌‌های بانک‌های اروپا شامل بیش از ۲۵ کلاس و همچنین زمان و مقدار آن‌هاست که در اینجا با 1 و 0 شماره گذاری شده اند.
ما در اینجا قصد داریم ابتدا با پیش پردازش و سپس با استفاده از الگوریتم \lr{kmeans} برای خوشه بندی و همچنین الگوریتم های \lr{SVC, LogisticRegression, KNeighborsClassifier} 
به پاسخی برای دسته بندی و همچنین خوشه بندی مجموعه داده برسیم.	
}


% کلمات کلیدی پایان‌نامه را وارد کنید
\keywords{
	\lr{classification}
	، هوش مصنوعی،
	\lr{clustering}
}



% \AUTtitle
%%%%%%%%%%%%%%%%%%%%%%%%%%%%%%%%%%
\vspace*{7cm}
\thispagestyle{empty}
\begin{center}
	\includegraphics[height=5cm,width=12cm]{Images/besm.jpg}
\end{center}