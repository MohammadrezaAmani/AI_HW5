\chapter{
	پیاده سازی
}

\section{مقدمه}

کد به زبان \lr{Python} نوشته شده است، بخش‌های کامل کد رو می‌توانید در  
\href{https://github.com/MohammadrezaAmani/AI_HW5}{گیتهاب}
مشاهده کنید.\\
کد به شکل کامل با استفاده از مفاهیم \lr{OOP} نوشته شده است. لیست کلاس ها شامل موارد زیر می‌باشد:\\


\section{پیش‌نیازها}

برای استفاده از کد ها نیاز به استفاده از زبان برنامه نویسی پایتون و کتابخانه‌های \lr{numpy, pandas, matplotlib, seaborn} و \lr{scikit-learn} دارید.
که با دستور 
\lr{pip install numpy pandas matplotlib seaborn scikit-learn}
میتوانید آن‌ها را نصب کنید.

\section{ساختار کلی کد}
کد به شکل OOP نوشته شده اما در آن متاسفانه زیبایی برنامه نویسی رعایت نشده که ان شاالله در آینده تغییر خواهد یافت.
کد از یک کلاس به نام \lr{CreditCardFraudClassifier} تشکیل شده است که وظیفه ی آن ایمپلیمنت کردن مباحث مورد نیاز برای کد است.

\lr{توابع}

\subsection{\lr{info}}
این تابع اطلاعات کلی از کد را نمایش می‌دهد.

\subsection{\lr{describe}}
مولفه های کد مانند کمینه مقدار و بیشینه مقدار و فیچر ها را نمایش می‌هد.
\subsection{\lr{process}}
کار های پیش پردازش دیتا بر عهده ی این تابع می‌باشد که شامل 
\\ 
بالانس کردن دیتا
\\ 
نورمال کردن مقادیر آن
\\ حذف مقادیر بی تاثیر مانند زمان
\\
جدا کردن داده به نسبت بیست به هشتاد برای رسیدن به پاسخ نهایی و تست آن 

\subsection{\lr{visualize\_kmeans\_clusters}}
این تابع با استفاده از روش خوشه بندی \lr{kmeans}
مجموعه داده را در دو مجموعه خوشه بندی کرده و نتیجه ی نهایی را نمایش می‌دهد.

\subsection{\lr{test\_all\_models}}

این تابع با دریافت لیستی از مدل ها تمام آن‌ها را تست کرده و نتایح را تحلیل و گزارش می‌کند.

